% camera ready
\section{Abstract}

In this report, we describe our experience participating in Hungry Geese, a Kaggle Simulation competition. In each game, four geese programmed by the teams will compete with each other. Similar to the Snake game made popular in Nokia phones, the geese extends its length by eating food and dies when its head bumps into any of the snakes. The objective of each goose is to survive 200 steps with the longest length.

Our approach to the competition is to improve on a publicly shared agent AlphaGeese. AlphaGeese follows the approach of AlphaGo, which uses a neural network model with Monte Carlo Tree Search. We have tried alternate model architectures for the neural network model, but none has been proven to be better than the neural network model published on Kaggle. We have finetuned the neural network model by training it against other top rule-based agents. We have also experimented with different ways of improving the Monte Carlo Tree Search.

By using fine-tuned agents and modifying the Monte Carlo Tree Search, we have produced agents that have higher mean rating than the original AlphaGeese agents. Tentatively, we have received a 70th with a rating of 1055.3. We have received a Kaggle Bronze medal for reaching the top 100.

% Many would have played a classic Snake game in Nokia phone in the past. It was a very popular game because it was the first mobile gaming experience for many people. There is a Kaggle competition called Hungry Geese which revolves around creating bots to play in a multiplayer snake-like game. Our group would participate in this competition by utilizing reinforcement learning model.

% The goal of our participation is to be able to obtain a high score within the competition amongst other submitted agents.

